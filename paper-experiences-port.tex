\subsection{Athletic education in a business school curriculum}
We have adapted the athletic software engineering approach to an introductory web applications programming course which serves as the entry \"core\" course for the Management of Information Systems (MIS) major within the Shidler College of Business. While this course shares many similar objectives, goals, and challenges with typical software engineering courses, there are some important differences. MIS students are unlikely headed for a software engineering career yet they likely will be involved in some way with the acquisition, development, maintenance or management of software systems and thus must be exposed to fundamental software engineering concepts. The majority of MIS students will have little to no programming skills prior to this first course and are unlikely to have another course in the future. Yet subsequent core courses such as systems analysis and design will assume they have robust programming skills. Hence the challenge of this course is to have absolute novices (1) acquire basic programming skills, (2) acquire skills and strategies for becoming efficient in all phases of planning, designing, programming, documenting, and testing web-based applications, (3) understand why MIS people need the aforementioned skills. All this must be accomplished within a single semester. 

Owing to the challenges listed previously, our interest in the athletic approach is less on limiting distraction and more on rapidly building competence, and even more importantly, building {\em confidence} in developing software. Lack of confidence in software development has been a common problem for our MIS students. This has led to a number of unpleasant consequences downstream such as poor performance in subsequent MIS courses, high incidence of cheating and avoiding programming tasks and projects (e.g. getting others to do the programming), and a lack of enthusiasm for MIS or dropping out of the major (i.e. \"if this is what MIS is about I don't want to do it\"). Over the years we have tried, with limited success, many different approaches to address this such as real-client projects, gamification, flipped classroom, etc. However our experiences over the past year indicates that the athletic approach is effective in rapidly building both competence and confidence in software development for extremely novice MIS students. Unexpectedly the athletic approach also seems to {\em generate enjoyment and enthusiasm} for building software. 

We apply athletic software engineering similarly as described earlier but with WODs that focus on basic programming concepts (e.g. loops, arrays, etc.) rather than particular development technologies or software engineering concepts. WODs are used to solidify the application of a particular programming concept after it has been introduced and experienced from in-class lab exercises. The WODs also do not persist for the entire class. After all the basic programming concepts are introduced, students switch to building full applications outside of class. We have found WODs unsuitable for developing skills and confidence in the design and implementation of applications of moderate complexity. Mostly this is because novice programmers have not built up enough development efficiency to implement a moderate complexity system within the limited time in class. The particular stumbling block seems to be testing and debugging. This is not surprising given their limited experience and training in this area. However we also found that WODs have been ineffective building students design skills. Novice students need time to plan, research, and experiment with various designs to find a workable solution. When under time pressure they are prone to trying seemingly random bits of code or copying code that appears useful in the hopes of quickly getting to a solution. 
    