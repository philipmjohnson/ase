\subsection{Athletic education in a business school curriculum}

Co-author Port adapted the athletic software engineering approach to an introductory web applications programming course to a Management of Information Systems (MIS) major. %While MIS students may not need the technical depth of a developer, they likely will be involved in some way with the acquisition, development, maintenance or management of software systems and thus must be exposed to fundamental software engineering concepts. 
%
The challenge of this single semester course is to educate absolute novices to (1)~acquire basic programming fluency, (2)~acquire skills and strategies for becoming efficient in all phases of software development, and (3)~understand why and where MIS people apply software development skills.
%
%While this course shares many similar objectives, goals, and challenges with typical software engineering courses, there are some important differences. For example, MIS students are unlikely headed for a software engineering career and as such they will not need a high degree of technical depth in software development. Yet they likely will be involved in some way with the acquisition, development, maintenance or management of software systems and thus must be exposed to fundamental software engineering concepts. The challenge of this course is to educate absolute novices to (1) acquire basic programming fluency, (2) acquire skills and strategies for becoming efficient in all phases of planning, designing, programming, documenting, and testing software applications, (3) understand why and where MIS people need the aforementioned. This must be accomplished all within a single semester. 
%
Owing to these challenges, our interest in the athletic approach lies in rapidly building competence and confidence in developing software to increase students' future performance in MIS courses.

%. Lack of confidence has been a serious problem for our MIS students. It has led to a number of unpleasant consequences downstream such as poor performance in MIS courses, a high incidence of cheating, and aversion to programming tasks and projects, e.g., getting others to do the programming. 

Our experiences over the past year with the athletic approach indicates that it is highly effective in rapidly building both competence and confidence in software development for extremely novice MIS students. Unexpectedly, the athletic approach also appears to {\em generate enjoyment and enthusiasm} for building software after competence and confidence is achieved. In particular, it fosters determination in getting software to work and elation when it does, rather than fear and despondence when it does not. Curiously the athletic approach also fosters greater collaboration and students do not feel in competition with each other. To the contrary, they feel the drive to excel and help each other understand the material and master the assignments.
%and get through difficulties in lab exercises, WODs, and assignments.
    
Through student comments and survey data, co-author Port summarizes student reaction to his adaptation of athletic software engineering as follows:
\begin{itemize} 
\item Students like the practice WODs and feel they learn a great deal by attempting them and  then watching a solution video.
\item Students do not like in-class WODs and are frustrated when they repeatedly DNF. %They do not like the ``speed-game" and highly constrained environment. 
However they eventually learn how to succeed and believe WODs are essential for building their programming competence. %They do not believe WODs should be eliminated from the class or replaced with more traditional, outside of class assignments.
\item Running WODs until students no longer DNF builds confidence and enthusiasm. After completion of the WODs, students felt ready to take on the challenge of building full applications with more complexity and less guidance. 
\item Compared to previous classes, students who experienced the athletic approach performed better and a higher percentage of students performed successfully in subsequent MIS courses that depend on development skills.  
%\item Fewer students are dropping MIS as a major and there has been a notable increase in the number of entering the major. 
%\item Students who were initially fearful of or dreaded programming had shifted their views ranging from neutral to enthusiastic. All students believe their development skills greatly improved, typically well beyond their expectations.
\item The athletic approach can be discouraging to some students. They are disappointed or worried when they find their approach is not as effective as examples or other students.    
\end{itemize} 