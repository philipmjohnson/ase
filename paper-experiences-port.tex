\subsection{Athletic education in a business school curriculum}
We have adapted the athletic software engineering approach to an introductory web applications programming course which serves as the entry \"core\" course for the Management of Information Systems (MIS) major within the Shidler College of Business. While this course shares many similar objectives, goals, and challenges with typical software engineering courses, there are some important differences. MIS students are unlikely headed for a software engineering career yet they likely will be involved in some way with the acquisition, development, maintenance or management of software systems and thus must be exposed to fundamental software engineering concepts. The majority of MIS students will have little to no programming skills prior to this first course and are unlikely to have another course in the future. Yet subsequent core courses such as systems analysis and design will assume they have robust programming skills. Hence the challenge of this course is to have absolute novices (1) acquire basic programming skills, (2) acquire skills and strategies for becoming efficient in all phases of planning, designing, programming, documenting, and testing web-based applications, (3) understand why MIS people need the aforementioned skills. All this must be accomplished within a single semester. 

Owing to the challenges listed previously, our interest in the athletic approach is less on limiting distraction and more on rapidly building competence, and even more importantly, building {\em confidence} in developing software. Lack of confidence in software development has been a common problem for our MIS students. This has led to a number of unpleasant consequences downstream such as poor performance in subsequent MIS courses, high incidence of cheating and avoiding programming tasks and projects (e.g. getting others to do the programming), and a lack of enthusiasm for MIS or dropping out of the major (i.e. \"if this is what MIS is about I don't want to do it\"). Over the years we have tried, with limited success, many different approaches to address this such as real-client projects, gamification, flipped classroom, etc. However our experiences over the past year indicates that the athletic approach is effective in rapidly building both competence and confidence in software development for extremely novice MIS students. Unexpectedly the athletic approach also seems to {\em generate enjoyment and enthusiasm} for building software. 

We apply athletic software engineering similarly as described earlier but with WODs that focus on basic programming concepts (e.g. loops, arrays, etc.) rather than particular development technologies or software engineering concepts. There are 7-9 WODs used to solidify the application of a particular programming concept after it has been introduced and experienced from in-class lab exercises. The WODs also do not persist for the entire class. They are primary used to rapidly build basic programming confidence. Practice WODs solutions are an excellent opportunity to introduce good programming practices, tools, and efficiency techniques that typically are not easily applied in a natural hands-on way the students can immediately apply and appreciate e.g. search and replace with regular expressions. The WODs continue until the majority of the class are not DNFing. From the experience over two classes this occurs after 5-6 WODs. It's important to have a few WODs after this point to enable students to recognize their improvement efforts to successfully complete a WOD. After all the basic programming concepts are introduced, students switch to building full applications outside of class. We have found WODs unsuitable for developing skills and confidence in the design and implementation of applications of moderate complexity. Mostly this is because novice programmers have not built up enough development efficiency to implement a moderate complexity system within the limited time in class. The particular stumbling block seems to be testing and debugging. This is not surprising given their limited experience and training in this area. However we also found that WODs have been ineffective building students design skills. Novice students need time to plan, research, and experiment with various designs to find a workable solution. When under time pressure they are prone to trying seemingly random bits of code or copying code that appears useful in the hopes of quickly getting to a solution. 

Throughout the class students are asked to BLOG about their experiences in various areas. This is essential in helping them identify their challenges and recognize their improvements. It is also a handy way to evaluate the effectiveness of the athletic approach. Their first essay is to summarize their experience in attempting two web page practice WODs. For each practice WOD they complete, they are asked describe how long it took to finish and what they learned (e.g. about HTML, CSS, Netbeans, LiveReload, or Chrome Developer Tools). For each WOD that they DNF�d, explain what happened and what was learned. If they did a WOD multiple times, report the times of all attempts and what was learned by doing them more than once. It is important that this essay is done prior to the first in class WOD as it sets a baseline they can compare with in their effectiveness of preparing for a successful WODs. After they experience two in-class WODs they are asked to write an essay on discussing their experiences in performing WODs. What worked well, what they stumbled on, what was slow. They describe how the practice WODd helped to prepare them and what they could have done to be better prepared. After the WODs end students are asked to respond to the following questions:
\begin{enumerate}

\item My programming skills have greatly improved.
\item I am enthusiastic about programming.
\item I can do the practice WODs without looking at the screencast solutions.
\item The labs have helped me to learn and I am able to complete them with confidence.
\item I feel ready to move to the next stage of complexity in programming (building small web applications).
\item I think the class can be improved to help my learning.
There are things that work really well my learning in this class.

\end{enumerate}
    
	To summarize the results of our experience with the athletic approach:
\begin{itemize} 
\item Students like the practice WODs and feel they learn a great deal by attempting them then watching a solution video.
\item Students do not like in-class WODs and are frustrated when they repeatedly DNF. They do not like the speed-game and highly constrained environment. However they eventually learn how to succeed and believe they are essential for building their programming competence. They do not believe they should be eliminated from the class or replaced with more traditional homework assignments.
\item Running WODs until students no longer DNF builds confidence and enthusiasm. After completion of the WODs, students felt ready to take on the challenge of building full applications with more complexity and less guidance. 
\item Compared to previous classes, students who experienced the athletic approach performed better and a higher percentage of students performed successfully in subsequent MIS courses that depend on development skill.  
\item Fewer students are dropping MIS as a major and there has been a notable increase in the number of entering the major. 
\item Students who were initially fearful of or dreaded programming had shifted their views ranging from neutral to enthusiastic. All students believe their development skills greatly improved, typically well beyond their expectations.
\item The athletic approach can be discouraging to some students. They are disappointed or worried when they find their approach is not as effective as examples or other students.    
\end{itemize} 