\subsection{Athletic education in a business school curriculum}
We have adapted the athletic software engineering approach to an introductory web applications programming course which serves as the entry course to the Management of Information Systems (MIS) major within the Shidler College of Business. While this course shares many similar objectives, goals, and challenges with typical software engineering courses, there are some important differences. For example MIS students are unlikely headed for a software engineering career and as such they will not need a high degree of technical depth in software development. Yet they likely will be involved in some way with the acquisition, development, maintenance or management of software systems and thus must be exposed to fundamental software engineering concepts. The majority of our MIS students will have little or no programming skills prior to this course, and they are unlikely to take another programming based course in the future. None the less, subsequent core courses such as systems analysis and design will assume they have programming competency. The challenge of this course is to educate absolute novices to (1) acquire basic programming fluency, (2) acquire skills and strategies for becoming efficient in all phases of planning, designing, programming, documenting, and testing software applications, (3) understand why and where MIS people need the aforementioned. This must be accomplished all within a single semester. 

Owing to these challenges, our interest in the athletic approach lies in rapidly building competence, and perhaps more importantly, having {\em confidence} in developing software. Lack of confidence has been a serious problem for our MIS students. It has led to a number of unpleasant consequences downstream such as poor performance MIS courses, a high incidence of cheating, and aversion to programming tasks and projects e.g. getting others to do the programming. This can lead some students to a lack of enthusiasm for MIS or dropping out of the major i.e. {\em "if this is what MIS is about I don't want to do it."} 

We have tried, with limited success, many of the approaches discussed in section \ref{Athletic software engineering in a nutshell} to address the competence/confidence  problem. Our experiences over the past year with the athletic approach indicates that it is highly effective in rapidly building both competence and confidence in software development for extremely novice MIS students. Unexpectedly the athletic approach also appears to {\em generate enjoyment and enthusiasm} for building software after competence and confidence is achieved. In particular, it fosters determination in getting software to work and elation when it does, rather than fear and despondence when it does not. Curiously the athletic approach also fosters greater collaboration and students do not feel in competition with each other. To the contrary, each feel the drive to excel and frequently help each other out to understand the material and get through difficulties in lab exercises, WODs, and assignments.

Our WODs focus on basic programming concepts (e.g. loops, arrays, etc.) rather than particular development technologies or software engineering concepts. There are 7-9 WODs used to solidify the application of a particular programming concept after it has been introduced and experienced from in-class lab exercises. WODs also do not persist for the entire class. They are primary used to rapidly build basic programming confidence. Practice WODs solutions are an excellent opportunity to introduce good programming practices, tools, and efficiency techniques in a natural hands-on way that students can immediately apply and appreciate e.g. using regular expressions to search and replace a pattern of variable names. WODs continue until the majority of the class are not DNFing. From the experience over two classes this occurs after 5-6 WODs. It's important to have a few WODs after this point to enable students to recognize and appreciate their improvement efforts to successfully complete a WOD. After   basic programming concepts are covered, students switch to designing and implementing full applications of moderate complexity. We have found WODs unsuitable for this stage of the class. This may be because novice programmers have not built up enough development efficiency to design and implement such applications within the limited time available in class. The particular stumbling blocks seem to be identifying a viable design and testing and debugging. This is not surprising given their limited experience and training in these areas. WODs do not appear to be effective in building students design skills. Perhaps as discussed earlier this is due to the more creative rather than mechanical nature of design. It appears that novice programmers need time to plan, research, and experiment with various designs to identify and understand a workable solution. When under time pressure they fall into the particularly difficult habit to break of trying random bits of code or searching online and copying code from previous work or examples in class that appear useful in the hopes of quickly getting to a solution. They rarely think about how this is ineffective or how they might work more efficiently and effectively. For most students this habit dissipates after they DNF on WODs a few times.

As prescribed by the athletic approach, throughout the class students are asked to BLOG about their experiences and progress. This is essential in helping them identify  challenges and recognize improvement. This is also a handy way to evaluate the effectiveness of the athletic approach. Their first essay is to summarize their experience in attempting two HTML/CSS practice WODs. For each practice WOD they are asked describe how long it took to finish and what they learned (e.g. about HTML, CSS, Netbeans, LiveReload, or Chrome Developer Tools). For each WOD that they DNF'd, explain what happened and what was learned. If they did a WOD multiple times, report the times of all attempts and what was learned by doing them more than once. It is important that this essay is done prior to the first in-class WOD to set a baseline they can compare with later in their effectiveness of preparing for a successful WOD. After they experience two subsequent in-class WODs they are asked to write an essay discussing their experiences in performing WODs. What worked well, what they stumbled on, what was ineffective. They describe how practice WODs help to prepare them and what they could have done to be better prepared. After the in-class WODs end, students are asked to evaluate themselves with the following questions:
\begin{enumerate}
\item My programming skills have greatly improved.
\item I am enthusiastic about programming.
\item I can do the practice WODs without looking at the screencast solutions.
\item The labs have helped me to learn and I am able to complete them with confidence.
\item I feel ready to move to the next stage of complexity in programming (building small web applications).
\item I think the class can be improved to help my learning.
\item There are things that work really well my learning in this class.
\end{enumerate}
    
To summarize the results of our experience with the athletic approach:
\begin{itemize} 
\item Students like the practice WODs and feel they learn a great deal by attempting them then watching a solution video.
\item Students do not like in-class WODs and are frustrated when they repeatedly DNF. They do not like the "speed-game" and highly constrained environment. However they eventually learn how to succeed and believe WODs are essential for building their programming competence. They do not believe WODs should be eliminated from the class or replaced with more traditional, outside of class assignments.
\item Running WODs until students no longer DNF builds confidence and enthusiasm. After completion of the WODs, students felt ready to take on the challenge of building full applications with more complexity and less guidance. 
\item Compared to previous classes, students who experienced the athletic approach performed better and a higher percentage of students performed successfully in subsequent MIS courses that depend on development skill.  
\item Fewer students are dropping MIS as a major and there has been a notable increase in the number of entering the major. 
\item Students who were initially fearful of or dreaded programming had shifted their views ranging from neutral to enthusiastic. All students believe their development skills greatly improved, typically well beyond their expectations.
\item The athletic approach can be discouraging to some students. They are disappointed or worried when they find their approach is not as effective as examples or other students.    
\end{itemize} 