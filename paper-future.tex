\section{The future of software engineering education}

One thing we can say almost with certainty about the future: technology will be more distracting than ever. If students are currently distracted by Facebook, Twitter, and YikYak, imagine the allure of 3D virtual reality environments such as the Oculus Rift.  One thing we can also say: it is unlikely that the future will bring some educational silver bullet that makes learning software engineering quick and easy. 

Based on our prior experience with serious game design \cite{csdl2-12-06}, we do not believe that ``gamification'' holds a major role in the future of software engineering education. Providing leaderboards, awarding badges, and so forth can increase engagement to some degree, but we do not believe it is capable by itself of incentivizing the sustained, focus attention required to assimilate complex software development skillsets. We believe gamification works best for learning relatively simple, conceptual knowledge.

Another possibility for the future is that expansion of programming competitions such as TopCoder into the realm of software engineering.  Such a development would be quite interesting as a way to provide visibility tohighly skilled software engineers, but such competitions are meant as an assessment vehicle, not for learning. 

``Bootcamps'' are a recent innovation for teaching coding skills; these courses require full-time residency and total commitment for six weeks to six months, and in many cases take complete novices and help them to become employable as web developers.   It seems quite plausible that this model could be extended to software engineering with positive results: after all, complete immersion in any activity for six weeks to six months should yield significant progress.  However, since most people cannot check out of their lives (and livelihood) for such time periods, software engineering bootcamps, if they arise, will be a niche approach applicable to only a few people.

Based upon our initial experiences, we believe an athletic pedagogy will find its place in the future of software engineering education as a way to help students efficiently acquire mastery of the mechanics of software engineering, and thus create additional temporal and mental space for the creative problem solving required in our discipline. This approach is still in its infancy and will certainly be refined and improved with additional experience. We invite software engineering educators who find this approach of interest to join with us to move it forward. 

