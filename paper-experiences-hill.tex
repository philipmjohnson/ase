\subsection{Athletic education in introductory programming}

Co-author Hill deployed yet another adaptation of athletic software engineering, this time oriented to introductory programming in CS~1 (python) and CS~2 (Java). The in-class, timed problems were assigned as homework if the students did not finish. However, to receive an A grade, the assignment must have been correctly completed during class. Grading preference was given for correctness over submitting on time.

The classes were not completely flipped, but instead blended traditional lectures, code demos to problems such as previous WOD or practice WOD exercises, WODs, and in-class project work time. The CS~1 class focused on intraprocedural programming (ifs, loops, functions, etc.) within a single file, whereas the CS~2 course focused on object-oriented design and higher level design concepts in creating programs with multiple files (fields, methods, inheritance, abstract classes, interfaces, etc.).

Anecdotal feedback from students was positive for the practice WODs---many students said they liked learning from the videos and would sometimes request additional videos and practice WODs to help them learn difficult concepts. One student complained the videos were too long and meandering, since they walked through the thought process and steps taken by the instructor to solve the problem, rather than just jumping straight to the answer.

In a current offering of the CS~1 course, we simplified the WODs and scheduled them to occur during class \emph{before} the practice videos are posted. Students still have a time limit on the practice WOD, but cannot see the video solution until after the in-class WOD. This forces students to do the practice WOD if they want to prepare for the in-class WOD, which is taken almost verbatim from the practice WOD. The practice WODs remain ungraded.

Anonymous student survey feedback from the courses was mixed. In the CS~1 course, 18 of the 25 students registered responded to the survey, with two-thirds preferring the athletic approach over a more traditional style. Unfortunately, in the CS~2 course, only 5 students responded to the survey, rendering the results uninterpretable.  Unlike Port's students, students in CS~1 and CS~2 complained that the competitive nature of the WODs discouraged collaborative learning. For example, one CS~1 student said: ``... it created a hostile environment where people were afraid to admit that they didn't understand course material outside of class. Also, it made peers less likely to help each other or provide advice.''  On the other hand, another students noted that the competitive nature spurred them to ``do additional work using resources outside of the class."

Students from both courses agreed that the athletic structure kept them focused, and really enjoyed the practice WODs:
``It was less stressful doing homework [practice WOD] because I knew that the homework was not graded. The homework was there solely to help me learn, and that absence of negative pressure allowed me to focus and concentrate more so than I usually do.''
