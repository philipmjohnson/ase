\subsection{Athletic education in introductory programming}

%{\em Emily writes about her experiences here.}

In 2015, Hill deployed a modified athletic curriculum for introductory programming in CS~1 (python) and CS~2 (Java). The in-class, timed problem was assigned as homework if the students did not finish. However, to receive an A grade, the assignment must have been correctly completed during class. Grading preference was given for correctness over submitting on time.
%
The classes were not completely flipped, but instead blended traditional lectures, code demos to problems such as previous WOD or practice WOD exercises, WODs, and in-class project work time.

\subsubsection{Experience}

Anecdotal feedback from students was positive for the practice WODs---many students said they liked learning from the videos and would sometimes request additional videos and practice WODs to help them learn difficult concepts. One student complained the videos were too long and meandering, since they walked through the thought process and steps taken by the instructor to solve the problem, rather than just jumping straight to the answer.

As an instructor, the videos provided additional content hours outside of class to further help struggling students. The downside is that creating the videos took additional prep time, but this time would be gained in future offerings of the course.

From a student learning perspective, the downside of the practice WODs is that students who ran out of time before the WOD would simply watch the video without trying it themselves first. They would be lulled into a false sense of confidence that would leave them ill-equipped to succeed on the WODs. Perhaps being more strict on enforcing the WOD time limits, or offering multiple WODs to demonstrate mastery of a skill would have prevented this.

In a current offering of the CS~1 course, we have simplified the WODs and scheduled them to occur during class \emph{before} the practice videos are posted. Students still have a time limit on the practice WOD, but cannot see the video solution until after the in-class WOD. This forces students to do the practice WOD if they want to prepare for the in-class WOD, which is taken almost verbatim from the practice WOD. The practice WODs remain ungraded.

%Drawbacks: students watched the PWOD videos instead of doing them first. New adjustment: not releasing practice videos until after a brief in-class quiz. Insufficient data to determine success of technique. PWODs remain ungraded, but in-class quiz pulled directly from them.

\begin{table}

\begin{center}
\begin{tabular}{lrlrl}
Lecture format & CS 1 & (\%) & CS 2 & (\%) \\
\hline
traditional lecture & 5 & (28\%)  & 3 & (60\%)\\
code demo & 16 & (89\%)  & 4 & (80\%)  \\
WOD &  6 & (33\%) & 0 & (0\%) \\
assignment work time & 7  & (39\%) & 5 & (100\%) \\
\hline
Total respondents &  18 &  & 5 &  \\

\end{tabular}
\end{center}
\caption{Preferred class formats for CS 1 \& 2 (students could select multiple lecture format types).}
\label{tab:format}
\end{table}

\subsubsection{Student Feedback}

%\subsubsubsection{Future Adjustments}

Anonymous student survey feedback from the courses was mixed. In the CS~1 course, 18 of the 25 students registered responded to the survey, with two-thirds preferring the athletic approach over a more traditional style. In contrast, of the 20 students registered for the CS~2 course, only 5 students responded to the survey, and 80\% preferred a more traditional approach. 
%(The lack of participation may in itself indicate their opinion.) 
These students had taken a version of the CS~1 course the prior semester, with the same instructor, but with a more traditional homework and lab structure. Table~\ref{tab:format} shows the type of class format students in each course preferred. Students tend to prefer code demos, which is the category of the practice WODs, but they prefer them during class time. The CS~2 students who had a different course format with the same instructor in the prior semester preferred more traditional lectures and in-class work time, which were familiar activities to them from CS~1.

Students in both courses complained that the competitive nature of the WODs discouraged collaborative learning. A CS~1 student said:

\begin{quote}
... it created a hostile environment where people were afraid to admit that they didn't understand course material outside of class. Also, it made peers less likely to help each other or provide advice.
\end{quote}

A CS~2 student felt:

\begin{quote}
... it wasn't realistic. In a true development team, there is more of a focus on collaboration and working together rather than a competitive focus.
\end{quote}

Whereas other students found that the competitive nature spurred them to ``do additional work using resources outside of the class."

When the students succeeded, it tended to create a significant confidence boost, although some students felt like success was only doing what they were ``supposed to" and felt more defeated when they couldn't complete the WODs.

Students from both courses agreed that the athletic structure kept them focused, and really enjoyed the practice WODs:

\begin{quote}
It was less stressful doing homework [practice WOD] because I knew that the homework was not graded. The homework was there solely to help me learn, and that absence of negative pressure allowed me to focus and concentrate more so than I usually do.
\end{quote}

\subsection{Discussion}

The observations from the introductory programming classes mirrored observations from the other experiences. Students focused on learning design and organization in a CS~2 class and an MIS course struggled with timed assignments, whereas students in a more traditional CS~1 or software engineering? course with small, concrete assignments had a better experience. \todo{Phil, how would you characterize your SE class? More short programming classes or more design?}
