\subsection{Athletic education in introductory programming}

%{\em Emily writes about her experiences here.}

In 2015, Hill deployed a modified athletic curriculum for introductory programming in CS~1 (python) and CS~2 (Java). The in-class, timed problem was assigned as homework

\todo{Describe variations.} WODs became homeworks, must complete in class for A level grade. Not totally flipped, only flipped on WOD day (like a combined quiz \& HW day).

\subsubsection{Experience}

Very positive response to videos -- most watched. Provided additional content hours outside of class time.

Drawbacks: students watched the PWOD videos instead of doing them first. New adjustment: not releasing practice videos until after a brief in-class quiz. Insufficient data to determine success of technique. PWODs remain ungraded, but in-class quiz pulled directly from them.

\begin{table}

\begin{center}
\begin{tabular}{lrlrl}
Lecture format & CS 1 & (\%) & CS 2 & (\%) \\
\hline
traditional lecture & 5 & (28\%)  & 3 & (50\%)\\
code demo & 16 & (89\%)  & 4 & (67\%)  \\
WoD &  6 & (33\%) & 0 & (0\%) \\
assignment work time & 7  & (39\%) & 5 & (83\%) \\
\hline
Total respondents &  18 &  & 6 &  \\

\end{tabular}
\end{center}
\caption{Preferred class formats for CS 1 \& 2 (students could select multiple lecture format types).}
\label{tab:format}
\end{table}

\subsubsection{Student Feedback}

