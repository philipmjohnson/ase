%\section{Introduction}

%Startups in Silicon Valley and elsewhere have recently focused on {\em disrupting} education with online platforms such as Coursera, EdX, and CodeAcademy.  What is less obvious is a parallel, implicit, and much better funded focus on {\em distracting} education through the rise of mobile applications intended to encourage multi-tasking and continuous partial attention. For example, one designer cites ``design for distraction'' as his Number 1 Principle \cite{Wroblewski2014}, and others write of the rise of the {\em attention economy}, a ``system that revolves primarily around paying, receiving and seeking what is most intrinsically limited and not replaceable by anything else, namely the attention of other human beings''\cite{Chamorro-Premuzic2014}. This focus on distraction and attention has reaped dubious benefits: a 2013 Internet Trends report found that people check their phones an average of 150 times per day \cite{Meeker2013}. 
%
%Why does this matter to the future of software engineering?  It matters because 
%
%Future software engineers are typically enthusiastic consumers of modern, mobile technology, and because this same technology is actively degrading their ability to do the focused, in-depth learning required to acquire advanced software engineering skills.  

In the past 10 years, much evidence has accumulated regarding the negative effects of multi-tasking and continous partial attention on learning. One study found that students who multi-tasked spent only 65\% of their time actively learning, that it took them longer to complete assignments, that they made more mistakes, decreased their ability to remember material later, and that they showed less ability to generalize the information they learned to new contexts \cite{Paul2013}. 

Traditional educational approaches to software engineering education consisting of in-class lectures, unsupervised homework assignments, and occasional project deadlines are fertile ground for the distraction economy.  
%Social applications such as Facebook and Twitter are designed to encourage continuous partial attention, but the impact of this distraction on deep learning is significant.  
What can we do to win back student attention and help them learn the skills they need for modern software engineering? 

We present our initial experiences with an ``athletic'' approach to software engineering education, which co-author Johnson developed in 2013 as an experiment in software engineering education, and which has been adapted by co-authors Port and Hill in their own courses. 
%
The athletic software engineering pedagogy is inspired by athletic training, where time is viewed as a constrained resource that forces athletes to avoid distractions.
% or continuous partial attention to the task at hand.
Can software engineering education be redesigned as an athletic endeavor, and will this lead to more efficient and effective learning among students? 
%\todo{We observe that...}

%observation that athletic training and subsequent competitions, in contrast to traditional software engineering education, view time as a constrained resource, and athletes generally do not suffer from distraction or continuous partial attention during training or competition.  

%Even the ``flow state'', revered among programmers as a rarely achieved kind of software development satori, is  commonplace and unexceptional among athletes during competition. Can software engineering education be redesigned as an athletic endeavor, and will this lead to more efficient and effective learning among students?  In this paper, we will present our findings and what they might imply for the future of software engineering education. 









 