\section{Athletic software engineering in a nutshell}

To understand what makes our approach ``athletic'', it helps to first consider more common approaches to software engineering education. 

{\em Lecture-based survey.}  One traditional software engineering educational model involves the study of an Introduction to Software Engineering textbook, with lectures presenting material from various chapters, and tests that assess the ability of students to define or manipulate concepts such as the ``Spiral Model'', ``White Box Testing'', ``Extreme Programming'', etc. This approach has the benefit of being both comprehensive and consistent in the way students each semester encounter the material. However, it treats software engineering material at a conceptual level and is the most susceptible to decreased learning through distraction.

{\em Project-based practicum.}  An alternative to lecture-based survey involves projects, often through solicitation of ```real-world'' application requirements from the surrounding community.  In this approach, students form teams and attempt to build software to satisfy their customer needs.  The project-based approach tends to produce more engagement among many students, although the experiences encountered by each group varies a great deal based upon their community sponsor. In addition, the experience of a student within a single group can vary, and it is difficult to guarantee or assess that all students in the group are gaining the same software engineering experiences. 

{\em Flipped classroom.}  A third approach is to ``flip'' or ``invert'' the classroom. In this case, lecture material is recorded and provided to students via YouTube, leaving class time for more active learning opportunities.  