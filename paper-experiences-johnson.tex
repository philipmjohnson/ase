\subsection{Athletic education in software engineering}

Co-author Johnson taught software engineering in an athletic style to both an undergraduate software engineering class in 2014 and a graduate software engineering class in 2015, with a total of 29 students. To assess the approach, he required students to write technical essays on their progress through the course and administered a questionnaire near the end of the semester that obtained opions from all students in both courses. A brief summary of the significant findings include the following:

{\em Students like the athletic approach.} Out of 29 students surveyed, all but one (96\%) prefer athletic software engineering to a more traditional course structure.  One student commented, ``I would choose to do WODs over the traditional approach because it helps you to become accustomed to working under pressure. I find myself learning more this way due to having to remember what I've done rather than searching up on how to do something and then forgetting soon after.''

{\em Students will redo training problems to gain skill mastery.}  While athletic software engineering makes it possible for students to repeat training problems if they do not achieve adequate performance, it does not mean students will do that in practice. %Think back to your own scholastic endeavors: did you ever redo a home assignment from scratch just to see if you could finish it faster? 
Surprisingly, the majority (72\%) of students found it useful to repeat the training problems, and most repeated over half of the training problems at least once. 

{\em The athletic approach improves focus.}  Most students (82\%) indicated that they believed that athletic software engineering helped improve their focus while learning the material.  One student commented, ``Like many students, when I do work at home I get distracted easily. [...] WODs definitely helped me to accomplish more in less time.''

{\em The athletic approach creates ``comfort under pressure''.}  Pressure is a part of a software developer's life.  Over 80\% of the students felt that this approach helped them to feel comfortable with programming under pressure.  %One student commented: ``I am indeed more confident in programming under pressure. I have learned to think not more quickly, but more calmly and collectively, as that is probably most important in completing a task faster.''

In summary, initial results from student self-assessment of athletic software engineering as implemented by co-author Johnson indicates they prefer this style of education, they are motivated to practice skills repetitively to improve their efficiency, they believe the pedagogy improves focus, and they acquire a level of comfort with pressure during programming tasks. 

%Let's now look at some adaptations of this approach, and the different kinds of outcomes that resulted.





